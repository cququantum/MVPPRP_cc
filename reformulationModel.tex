% This document is compiled using pdfLaTeX
% You can switch XeLaTeX/pdfLaTeX/LaTeX/LuaLaTeX in Settings

\documentclass{article}
\usepackage{amsmath,amssymb,amsfonts,amsthm}
\usepackage{enumitem}
\usepackage{geometry}
\usepackage{mathrsfs}
\geometry{a4paper, margin=1in}
\usepackage{ctex} % 支持中文
\usepackage{hyperref} % 启用超链接功能
\usepackage{float}
\usepackage[linesnumbered,ruled,vlined]{algorithm2e}

\newtheorem{assumption}{假设}
\newtheorem{remark}{注释}
\newtheorem{definition}{定义}

\begin{document}

\section{2.4 模型改写}

\subsection{改写思路}
引入基于最短路径的建模方法。主要改写内容:
\begin{itemize}
\item 引入虚拟期$l+1$,定义扩展时间集合$T' = T \cup \{l+1\}$
\item 引入参数$g_{ivt}$表示收集点$i$在第$t$期被访问且上次访问在第$v$期时的总取货量
\item 引入参数$e_{ivt}$表示相应的库存持有成本
\item 引入变量$\lambda_{ivt}$表示访问序列的二进制决策变量
\item 将库存管理约束转化为最短路径约束
\end{itemize}



\newpage
\noindent
扩展集合定义 $T' = T \cup \{l+1\}$ \\
定义$g_{ivt}$为收集点$i$在第$t$期被访问,上次访问在第$v$期时的总取货量,访问后库存清零:
\begin{equation*}
g_{ivt} = \begin{cases}
I_{i0} + \sum_{j=1}^{t} s_{ij}, & \text{如果 } v = 0 \text{(首次访问)}\\
\sum_{j=v+1}^{t} s_{ij}, & \text{如果 } 0 < v < t \leq l+1
\end{cases}
\end{equation*}
定义$e_{ivt}$为收集点$i$在第$t$期被访问,上次访问在第$v$期时,从第$v$期到第$t-1$期末的库存成本:
\begin{equation*}
e_{ivt} = \begin{cases}
h_i \sum_{j=1}^{t-1} \left(I_{i0} + \sum_{r=1}^{j} s_{ir}\right), & \text{如果 } v = 0 \\
h_i \sum_{j=v+1}^{t-1} \left(\sum_{r=v+1}^{j} s_{ir}\right), & \text{如果 } 0 < v < t \leq l+1
\end{cases}
\end{equation*}
定义$\mu(i,t)$为收集点$i$在第$t$期访问后,下次最晚可以被访问的时期(不超过容量):
\[\mu(i,t) = \arg\max_{t<v\leq l+1}\{v : g_{itv} \leq L_i\} \]
定义 $\pi(i,t)$为收集点$i$若要在第$t$期访问,上次访问的最早可行时期:
\[\pi(i,t) = \arg\min_{0\leq v<t}\{v : g_{ivt} \leq L_i\} \]
新增决策变量$\lambda_{ivt} \in \{0,1\}$:如果收集点$i$在第$t$期被访问且上次访问在第$v$期,则为1


\subsection{改写后的目标函数}
\begin{equation}
\min Z = \sum_{t \in T} \left\{
\underbrace{\sum_{(i,j) \in A} c_{ij} x_{ijt}}_{\text{取货路由成本}}
+ \underbrace{u p_t + f y_t}_{\text{生产成本}}
+ \underbrace{h_0 I_{0t}}_{\text{工厂原料库存成本}}
+ \underbrace{h_p P_{0t}}_{\text{成品库存成本}}
\right\}
+ \underbrace{\sum_{t \in T'} \sum_{i \in V_S} \sum_{v=\pi(i,t)}^{t-1} e_{ivt}\lambda_{ivt}}_{\text{供应商库存成本}}
\end{equation}
% 修正说明:生产成本中的 u 参照论文参数表2.2 (page 22) 修改为 U

\subsection{改写后的约束条件}

\subsubsection{生产与成品库存约束}
\vspace{-1.5ex}
\begin{align}
P_{0,t-1} + p_t &= d_t + P_{0t}, && \forall t \in T && \text{(成品库存平衡,$t=1$时$P_{0,0}=P_{00}$)} \\
p_t &\leq C y_t, && \forall t \in T && \text{(生产能力限制)} \\
P_{0t} &\leq L_p, && \forall t \in T && \text{(成品库存容量)}
\end{align}

\subsubsection{工厂原料库存约束}
\vspace{-1.5ex}
\begin{align}
I_{0,t-1} + \sum_{i \in V_S} \sum_{v=\pi(i,t)}^{t-1} g_{ivt}\lambda_{ivt} &= k p_t + I_{0t}, && \forall t \in T && \text{(工厂库存平衡)} \\
I_{0t} &\leq L_0, && \forall t \in T && \text{(工厂库存容量)}
\end{align}
% 修正说明:调整约束(2-40) (2-41)的顺序,使其在(2-39)之后,匹配论文page 28-29的顺序

\subsubsection{收集点访问与库存约束}
\vspace{-1.5ex}
\begin{align}
\sum_{v=\pi(i,t)}^{t-1} \lambda_{ivt} &= z_{it}, && \forall i \in V_S, t \in T && \text{(链接$z$和$\lambda$变量)} \\
\sum_{t=1}^{\mu(i,0)} \lambda_{i0t} &= 1, && \forall i \in V_S && \text{(从初始状态(v=0)开始)} \\
\sum_{v=\pi(i,t)}^{t-1} \lambda_{ivt} - \sum_{\tau=t+1}^{\mu(i,t)} \lambda_{it\tau} &= 0, && \forall i \in V_S, t \in T && \text{(流平衡约束)} \\
\sum_{t=\pi(i,l+1)}^{l} \lambda_{it,l+1} &= 1, && \forall i \in V_S && \text{(到达终止状态(t=l+1))}
\end{align}
% 修正说明:
% 1. 调整约束(2-42)-(2-45)的顺序,使其在(2-41)之后
% 2. 论文(2-45)中 \lambda_{il,l+1} 应为 \lambda_{it,l+1} (t为求和索引),保留逻辑正确的版本

\subsubsection{车辆路径约束}
\vspace{-1.5ex}
\begin{align}
\sum_{i \in V_S} \sum_{v=\pi(i,t)}^{t-1} g_{ivt}\lambda_{ivt} &\leq Q m_t, && \forall t \in T && \text{(车辆总容量限制)} \\
\sum_{j \in V} x_{ijt} &= \sum_{j \in V} x_{jit}, && \forall i \in V_S, t \in T && \text{(流量平衡约束)} \\
\sum_{j \in V_S} x_{0jt} &= m_t, && \forall t \in T && \text{(车辆从工厂出发)} \\
\sum_{i \in V_S} x_{i,n+1,t} &= m_t, && \forall t \in T && \text{(车辆返回工厂)} \\
% 修正说明:论文(2-49) (page 29) 返回的节点是 n+1
\sum_{i \in V} x_{ijt} &= z_{jt}, && \forall j \in V_S, t \in T && \text{(收集点访问约束)} \\
m_t &\leq K, && \forall t \in T && \text{(车辆数量限制)} \\
% 修正说明:调整(2-51)顺序到(2-50)之后
u_{jt} - u_{it} + M(1-x_{ijt}) &\geq \sum_{v=\pi(i,t)}^{t-1} g_{ivt}\lambda_{ivt}, && \forall (i,j) \in A, t \in T && \text{(MTZ子回路消除)} \\
% 修正说明:论文(2-52) (page 29) 使用 M, 不是 Q; 右侧是节点 i 的需求, 不是 j; 范围是 \forall (i,j) \in A
u_{it} &\leq Q, && \forall i \in V, t \in T && \text{(载重上限约束)} \\
% 修正说明:论文(2-53) (page 30) 仅为 u_it <= Q, 范围是 \forall i \in V
u_{0t} &= 0, && \forall t \in T && \text{(工厂载重为零)}
\end{align}

\subsubsection{变量定义域}
\vspace{-1.5ex}
% 修正说明:调整了整个定义域块的顺序以匹配论文 (2-55) 到 (2-63)
\begin{align}
y_t &\in \{0,1\}, && \forall t \in T \\
z_{it} &\in \{0,1\}, && \forall i \in V_S, t \in T \\
\lambda_{ivt} &\in \{0,1\}, && \forall i \in V_S, v,t \in T', v < t \\
x_{ijt} &\in \{0,1\}, && \forall (i,j) \in A, t \in T \\
m_t &\in \{0,1,\ldots,K\}, && \forall t \in T \\
p_t &\geq 0, && \forall t \in T \\
P_{0t} &\geq 0, && \forall t \in T \\
I_{0t} &\geq 0, && \forall t \in T \\
u_{it} &\geq 0, && \forall i \in V, t \in T
\end{align}

\begin{itemize}
    \item $y_t $:时期 $t$ 是否进行生产
    \item $p_t$:时期 $t$ 的成品生产量
    \item $P_{0t}$:时期 $t$ 结束时的成品库存
    \item $I_{0t}$:时期 $t$ 结束时工厂的原料库存
    \item $I_{it}$:时期 $t$ 结束时收集点 $i$ 的库存
    \item $z_{it} $:时期 $t$ 是否访问收集点 $i$
    \item $q_{it}$:时期 $t$ 从收集点 $i$ 的取货量
    \item $m_t$:时期 $t$ 使用的车辆数
    \item $x_{ijt} $:时期 $t$ 是否使用弧 $(i,j)$
    \item $u_{it}$:时期 $t$ 车辆到达节点 $i$ 时的累计载重
    \item $d_t$:时期 $t$ 的成品交付需求量
\end{itemize}

\end{document}
