% This document is compiled using pdfLaTeX
% You can switch XeLaTeX/pdfLaTeX/LaTeX/LuaLaTeX in Settings

\documentclass{article}
\usepackage{amsmath,amssymb,amsfonts,amsthm}
\usepackage{enumitem}
\usepackage{geometry}
\usepackage{mathrsfs}
\geometry{a4paper, margin=1in}
\usepackage{ctex} % 支持中文
\usepackage{hyperref} % 启用超链接功能
\usepackage{float}
\usepackage[linesnumbered,ruled,vlined]{algorithm2e}

\newtheorem{assumption}{假设}
\newtheorem{remark}{注释}
\newtheorem{definition}{定义}

\begin{document}

\maketitle

% ============================================================
%  LBBD additions (v3): Fix omega_t definition and its link to
%  original routing-cost term sum c_ij x_ijt.
%  Usage: \input{lbbd_vrp_cost_v3.tex}
% ============================================================

\section{车辆路径成本的逻辑Benders分解(LBBD)---模型增补(含$\omega_t$定义)}

% ------------------------------------------------------------
% A. BMP edits (replace routing cost by omega; remove routing vars)
% ------------------------------------------------------------
\subsection{Benders主问题(BMP)需要新增/替换的部分}

\paragraph{A.1 新增变量(每期路径成本变量)}
\begin{align}
\omega_t \ge 0,\qquad \forall t\in T .
\end{align}

\paragraph{A.2 $\omega_t$与原始路由成本$\sum c_{ij}x_{ijt}$的关系(必须明确给实现用)}
在\textbf{完整模型}中,时期$t$的车辆路径成本为
\begin{equation}
\text{RouteCost}_t(x)=\sum_{(i,j)\in A} c_{ij}\,x_{ijt}.
\label{eq:routecost_arc}
\end{equation}
在LBBD中,将该成本视为一个\textbf{子问题最优值}并用变量$\omega_t$表示:
\begin{equation}
\omega_t\ :=\ \min\Bigl\{\ \sum_{(i,j)\in A} c_{ij}\,x_{ijt}\ :\ (x_{\cdot\cdot t},\text{其余路由变量})\ \text{满足时期}t\text{的CVRP约束,且服务集合/需求由}(z,\lambda)\text{决定}\ \Bigr\}.
\label{eq:omega_def_arc}
\end{equation}
因此,BMP里不再显式出现$x_{ijt}$,而是通过Benders割逐步保证$\omega_t$不低估\eqref{eq:omega_def_arc}中的最优值。

\paragraph{A.3 目标函数中的替换}
将你现有(原始模型或改写后模型)目标函数中的车辆路径成本项
\[
\sum_{t\in T}\ \sum_{(i,j)\in A} c_{ij}\,x_{ijt}
\]
替换为
\[
\sum_{t\in T}\omega_t .
\]
(生产/库存/收集点库存持有等其它成本项保持不变。)

\paragraph{A.4 从BMP中移除的车辆路径变量与约束(全部转入BSP)}
在BMP中移除车辆路径相关变量与约束:
\[
x_{ijt},\ u_{it},\ m_t\ (\text{若你在模型中仍保留}),\ \text{及所有路由约束(流平衡/出入库/车辆数/容量/MTZ等)}.
\]
BMP仅保留生产/库存/访问序列($z,\lambda$)等相关约束(即第2章原始/改写后模型中除路由部分以外的约束)。

% ------------------------------------------------------------
% B. BSP definition: CVRP(t) in set-partitioning form (equivalent cost)
% ------------------------------------------------------------
\subsection{Benders子问题(BSP):给定$(z,\lambda)$计算精确路径成本$\phi(z,\lambda)$}

给定BMP整数解$(\bar z,\bar\lambda)$,子问题BSP定义为
\begin{equation}
\phi(\bar z,\bar\lambda)=\sum_{t\in T}\omega_t^*,
\end{equation}
其中$\omega_t^*$为时期$t$的CVRP$(t)$最优值。若任一时期CVRP$(t)$不可行,则$\phi(\bar z,\bar\lambda)=+\infty$。

\paragraph{B.1 用集合划分形式表示CVRP$(t)$(并与$\sum c_{ij}x_{ijt}$等价)}
对时期$t$,由$(\bar z,\bar\lambda)$确定该期必须服务的客户集合与需求。
令$\mathcal R^t$为所有容量可行路线集合(每条路线从工厂出发并返回工厂),
$b_r$为路线$r$的行驶成本,定义为
\begin{equation}
b_r\ :=\ \sum_{(i,j)\in r} c_{ij},
\label{eq:br_def}
\end{equation}
$a_{ir}=1$表示路线$r$访问客户$i$,否则0;$\xi_r$为是否选用路线$r$的二元变量。
则CVRP$(t)$可写为:
\begin{align}
\text{CVRP}(t):\quad
\omega_t^*=\min\ & \sum_{r\in\mathcal R^t} b_r\,\xi_r \notag\\
\text{s.t.}\quad
& \sum_{r\in\mathcal R^t} a_{ir}\,\xi_r = 1, && \forall i\in N_c^t(\bar z) \notag\\
& \sum_{r\in\mathcal R^t} \xi_r \le K, \notag\\
& \xi_r\in\{0,1\}, && \forall r\in\mathcal R^t .
\label{eq:cvrp_sp_t}
\end{align}
在该定义下,\eqref{eq:cvrp_sp_t}的最优值与\eqref{eq:omega_def_arc}(弧变量形式)的最优值一致,
即“$\omega_t$就是原始$\sum c_{ij}x_{ijt}$在该期最优路由下的取值”。

% ------------------------------------------------------------
% C. Core LBBD cuts used in Section 3.2.2 (f_lambda, g)
% ------------------------------------------------------------
\subsection{用于LBBD回调的核心割(对应第3章3.2.2)}

\paragraph{C.1 Hamming距离函数 $f_{\bar\lambda}(\lambda)$(式(3-10))}
对当前BMP整数解$\bar\lambda$,定义
\begin{align}
f_{\bar\lambda}(\lambda)
=&\sum_{t\in T}\ \sum_{i\in V_S}\ \sum_{v=\pi(i,t)}^{t-1}\Bigl[
\mathbf{1}(\bar\lambda_{ivt}=1)\,(1-\lambda_{ivt})
+
\mathbf{1}(\bar\lambda_{ivt}=0)\,\lambda_{ivt}
\Bigr]. \label{eq:flambda_def}
\end{align}
实现时建议构造两类索引集合并展开:
\[
f_{\bar\lambda}(\lambda)=
\sum_{(i,v,t)\in\mathcal I^1(\bar\lambda)} (1-\lambda_{ivt})
+
\sum_{(i,v,t)\in\mathcal I^0(\bar\lambda)} \lambda_{ivt},
\]
其中
$\mathcal I^1(\bar\lambda)=\{(i,v,t):\bar\lambda_{ivt}=1\}$,
$\mathcal I^0(\bar\lambda)=\{(i,v,t):\bar\lambda_{ivt}=0\}$(索引范围均为$t\in T,i\in V_S,v=\pi(i,t),\ldots,t-1$)。

\paragraph{C.2 不可行性割(No-good cut,式(3-9))}
当BSP在$(\bar z,\bar\lambda)$下不可行时,向BMP添加:
\begin{equation}
f_{\bar\lambda}(\lambda)\ \ge\ 1.
\label{eq:nogood_feas_39}
\end{equation}

\paragraph{C.3 最优性割}
当BSP可行,且$\phi(\bar z,\bar\lambda)>\sum_{t\in T}\bar\omega_t$时,
向BMP添加:
\begin{equation}
\sum_{t\in T}\omega_t\ \ge\ g_{(\bar z,\bar\lambda)}(\lambda),
\label{eq:opt_g_311}
\end{equation}
其中
\begin{equation}
g_{(\bar z,\bar\lambda)}(\lambda)
=\phi(\bar z,\bar\lambda)
-\bigl(\phi(\bar z,\bar\lambda)\bigr)\,f_{\bar\lambda}(\lambda).
\label{eq:g_def_312}
\end{equation}

\paragraph{C.4 直接可加到BMP的线性形式(推荐实现用)}
\begin{equation}
\sum_{t\in T}\omega_t
+
\bigl(\phi(\bar z,\bar\lambda)\bigr)\,f_{\bar\lambda}(\lambda)
\ \ge\
\phi(\bar z,\bar\lambda).
\label{eq:opt_g_linear}
\end{equation}

% ------------------------------------------------------------
% D. Stronger feasibility cuts (Section 3.3.2): (3-14)(3-15)
% ------------------------------------------------------------
\subsection{加强可行性割}

当存在时期$t$使得CVRP$(t)$不可行,令
\[
N_c^t(\bar z)=\{\,i\in V_S:\ \bar z_{it}=1\,\},
\quad
\bar v_i\ \text{满足}\ \bar\lambda_{i\bar v_i t}=1.
\]
以下可行性割均针对当前BMP整数解触发的某一具体不可行时期 $t$ 构造并添加(即按触发时期逐条生成),
不表示对所有时期同时使用同一组系数。

\paragraph{D.1 基础可行性割(式(3-14))}
\begin{equation}
\sum_{i\in N_c^t(\bar z)}\Bigl(1-\lambda_{i\bar v_i t}\Bigr)
+
\sum_{i\in N_c^t(\bar z)}\ \sum_{\substack{v=\pi(i,t)\\ \bar\lambda_{ivt}=0}}^{t-1}\lambda_{ivt}
\ \ge\ 1.
\label{eq:feas_314}
\end{equation}

\paragraph{D.2 强可行性割(式(3-15))}
\begin{equation}
\sum_{i\in N_c^t(\bar z)}\Bigl(1-\lambda_{i\bar v_i t}\Bigr)\ \ge\ 1.
\label{eq:feas_315}
\end{equation}

% ------------------------------------------------------------
% E. Dual-based optimality cuts (Section 3.3.3): (3-18)(3-19)
% ------------------------------------------------------------
\subsection{对偶最优性割(对应第3章3.3.3:式(3-18))}

CVRP$(t)$的LP松弛对偶可写为:
\begin{align}
\text{DualCVRP}(t):\quad
\max\ & \sum_{i\in N_c^t(\bar z)} u_i + K u_0 \notag\\
\text{s.t.}\quad
& \sum_{i\in N_c^t(\bar z)} a_{ir} u_i + u_0 \le b_r, && \forall r\in\mathcal R^t \notag\\
& u_i\in\mathbb R, && \forall i\in N_c^t(\bar z), \notag\\
& u_0\le 0 .
\label{eq:dual_cvrp_t}
\end{align}

设\eqref{eq:dual_cvrp_t}的最优解为$(u_i^*,u_0^*)$,则可向BMP添加:

其中 $(u_i^*,u_0^*)$ 来自当前BMP整数解下某一具体时期 $t$ 的子问题(或其LP松弛)对偶最优解,
故该割按时期逐条生成。

\paragraph{E.1 最优性割(式(3-18))}
\begin{equation}
\omega_t\ \ge\ \sum_{i\in N_c^t(\bar z)} u_i^*\cdot \lambda_{i\bar v_i t}\ +\ K u_0^*.
\label{eq:opt_318}
\end{equation}

% ------------------------------------------------------------
% G. Implementable LBBD algorithm skeleton
% ------------------------------------------------------------
\subsection{LBBD迭代算法(与第3章一致的可编码伪代码)}

\begin{algorithm}[H]
\caption{LBBD:仅分解车辆路径成本(与论文3.2.2三分支一致)}
\DontPrintSemicolon
\KwIn{参数与集合;车辆容量$Q$;车辆上限$K$;规划期$T=\{1,\dots,l\}$;容差$\varepsilon$}
\KwOut{最优解与最优值}

\textbf{Step 1 (初始化BMP):} 引入$\omega_t\ge0$并移除路由变量/约束;割集$\mathcal C\leftarrow\emptyset$\;
$UB\leftarrow+\infty$\;

\While{true}{
  \textbf{(1) 解BMP到整数最优}\;
  得到$(\bar z,\bar\lambda,\bar\omega)$,并令$LB \leftarrow$ 当前BMP目标值\;

  \textbf{(2) 解BSP(逐期CVRP)}\;
  $\textit{infeasible}\leftarrow false$,$\phi\leftarrow 0$\;
  \ForEach{$t\in T$}{
    由$(\bar z,\bar\lambda)$构造时期$t$的客户集合$N_c^t=\{i:\bar z_{it}=1\}$与需求$q_{it}$\;
    求解$\text{CVRP}(t)$(或先用BPP做可行性判定)得到$\omega_t^*$;\;
    \If{$\text{CVRP}(t)$不可行}{
      $\textit{infeasible}\leftarrow true$,记录该时期$t$\;
      \textbf{break}\;
    }
    $\phi \leftarrow \phi + \omega_t^*$\;
  }

  \textbf{(3) 若BSP不可行:加可行性割并继续迭代}\;
  \If{$\textit{infeasible}$}{
    加可行性割:优先用强可行性割(3-15);或用(3-14)/(3-9)\;
    $\mathcal C\leftarrow \mathcal C\cup\{\text{新割}\}$\;
    \textbf{continue}\;
  }

  \textbf{(4) BSP可行:更新上界}\;
  $UB\leftarrow \min\{UB,\ C_{\text{nonVRP}}(\bar z,\bar\lambda)+\phi\}$\;

  \textbf{(5) 若需要:加最优性割}\;
  \If{$\phi > \sum_{t\in T}\bar\omega_t + \varepsilon$}{
     加组合最优性割(3-11)(3-12);或加对偶割(3-19)\;
     $\mathcal C\leftarrow \mathcal C\cup\{\text{新割}\}$\;
     \textbf{continue}\;
  }

  \textbf{(6) 收敛判定}\;
  \If{$UB-LB\le \varepsilon$}{\textbf{break}}
}
\end{algorithm}

\end{document}