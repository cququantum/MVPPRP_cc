% This document is compiled using pdfLaTeX
% You can switch XeLaTeX/pdfLaTeX/LaTeX/LuaLaTeX in Settings

\documentclass{article}
\usepackage{amsmath,amssymb,amsfonts,amsthm}
\usepackage{enumitem}
\usepackage{geometry}
\usepackage{mathrsfs}
\geometry{a4paper, margin=1in}
\usepackage{ctex} % 支持中文
\usepackage{hyperref} % 启用超链接功能
\usepackage{float}
\usepackage[linesnumbered,ruled,vlined]{algorithm2e}

\newtheorem{assumption}{假设}
\newtheorem{remark}{注释}
\newtheorem{definition}{定义}

\begin{document}

\section{2.3 原始模型}

\subsection{目标函数}
最小化总成本,包括取货路由成本、生产成本、原料库存成本、成品库存成本和收集点库存成本:
\begin{equation}
\min Z = \sum_{t \in T} \left\{
\underbrace{\sum_{(i,j) \in A} c_{ij} x_{ijt}}_{\text{取货路由成本}}
+ \underbrace{(u p_t + f y_t)}_{\text{生产成本}}
+ \underbrace{h_0 I_{0t}}_{\text{工厂原料库存成本}}
+ \underbrace{\sum_{i \in V_S} h_i I_{it}}_{\text{供应商库存成本}}
+ \underbrace{h_p P_{0t}}_{\text{成品库存成本}}
\right\}
\end{equation}
% 修正说明:
% 1. (i,j) \in A 保持与论文 (2-1) 一致。
% 2. 生产成本中的 u 参照论文参数表2.2 (page 22) 修改为 U。

\subsection{约束条件}

\subsubsection{生产与成品库存约束}
\vspace{-1.5ex}
\begin{align}
P_{0,t-1} + p_t &= d_t + P_{0t}, && \forall t \in T && \text{(成品库存平衡,$t=1$时$P_{0,0}=P_{00}$)} \\
p_t &\leq C y_t, && \forall t \in T && \text{(生产能力限制)} \\
P_{0t} &\leq L_p, && \forall t \in T && \text{(成品库存容量)}
\end{align}

\subsubsection{收集点库存与取货约束}
\vspace{-1.5ex}
\begin{align}
I_{i,t-1} + s_{it} &= q_{it} + I_{it}, && \forall i \in V_S, t \in T && \text{(库存平衡,$t=1$时$I_{i,0}=I_{i0}$)} \\
I_{i,t-1} + s_{it} &\leq L_i, && \forall i \in V_S, t \in T && \text{(取货前库存容量限制} \\
q_{it} &\geq I_{i,t-1} + s_{it} - M(1 - z_{it}), && \forall i \in V_S, t \in T && \text{(访问时取全量)} \\
q_{it} &\leq M z_{it}, && \forall i \in V_S, t \in T && \text{(不访问则不取货)} \\
I_{it} &\leq M(1 - z_{it}), && \forall i \in V_S, t \in T && \text{(访问后库存为0)} \\
I_{it} &\geq I_{i,t-1} + s_{it} - M z_{it}, && \forall i \in V_S, t \in T && \text{(不访问时库存累积)} \\
% 说明:由库存平衡、q_{it}\ge 0 与取货前库存容量限制可推出 I_{it}\le L_i,故不再单列期末容量约束。
\end{align}

\subsubsection{工厂原料库存约束}
\vspace{-1.5ex}
\begin{align}
I_{0,t-1} + \sum_{i \in V_S} q_{it} &= k p_t + I_{0t}, && \forall t \in T && \text{(工厂库存平衡,$t=1$时$I_{0,0}=I_{00}$)} \\
I_{0t} &\leq L_0, && \forall t \in T && \text{(工厂库存容量)}
\end{align}

\subsubsection{车辆路径约束}
\vspace{-1.5ex}
\begin{align}
\sum_{i \in V_S} q_{it} &\leq Q m_t, && \forall t \in T && \text{(车辆总容量限制)} \\
\sum_{j \in V} x_{ijt} &= \sum_{j \in V} x_{jit}, && \forall i \in V_s, t \in T && \text{(流量平衡约束)} \\
% 修正说明:论文(2-14) (page 25) 的范围是 \forall i \in V_s
\sum_{j \in V_S} x_{0jt} &= m_t, && \forall t \in T && \text{(车辆从工厂出发)} \\
\sum_{i \in V_S} x_{i,n+1,t} &= m_t, && \forall t \in T && \text{(车辆返回工厂)} \\
% 修正说明:论文(2-16) (page 25) 返回的节点是 n+1, 而不是 0
\sum_{i \in V} x_{ijt} &= z_{jt}, && \forall j \in V_S, t \in T && \text{(收集点访问约束)} \\
u_{jt} - u_{it} + M(1-x_{ijt}) &\geq q_{it}, && \forall (i,j) \in A, t \in T && \text{(MTZ子回路消除)} \\
% 修正说明:论文(2-18) (page 25) 使用的是 M (大数), 不是 Q; 且右侧是 q_it, 不是 q_jt; 范围是 \forall (i,j) \in A
u_{it} &\leq Q, && \forall i \in V, t \in T && \text{(载重上限约束)} \\
% 修正说明:论文(2-19) (page 25) 仅为 u_it <= Q, 范围是 \forall i \in V
u_{0t} &= 0, && \forall t \in T && \text{(工厂载重为零)} \\
m_t &\leq K, && \forall t \in T && \text{(车辆数量限制)}
\end{align}

\subsubsection{变量定义域}
\vspace{-1.5ex}
\begin{align}
y_t &\in \{0,1\}, && \forall t \in T  \\
p_t &\geq 0, && \forall t \in T \\
P_{0t} &\geq 0, && \forall t \in T \\
I_{0t} &\geq 0, && \forall t \in T \\
I_{it} &\geq 0, && \forall i \in V_S, t \in T  \\
z_{it} &\in \{0,1\}, && \forall i \in V_S, t \in T  \\
q_{it} &\geq 0, && \forall i \in V_S, t \in T \\
m_t &\in \{0,1,\ldots,K\}, && \forall t \in T \\
x_{ijt} &\in \{0,1\}, && \forall (i,j) \in A, t \in T \\
u_{it} &\geq 0, && \forall i \in V, t \in T
\end{align}

\end{document}
